\chapter{Ողջո՜ւյն, աշխա՛րհ}

Մի որևէ ծրագրավորման լեզվի ուսումնասիրությունը սկսելիս 
առաջին ծրագիրն ընտրում են այնպես, որ այն չպարունակի բարդ 
լեզվական ու ալգորիթմական կառուցվածքներ, բայց սկսնակ 
ծրագրավորողին հնարավորություն տա ամբողջությամբ տեսնել մեկ 
ծրագրի կյանքը՝ ստեղծումից մինչև կատարում։ C++ լեզվով գրված 
մեր առաջին ծրագիրն էլ բարդ չէ․ այն արտածման ստանդարտ հոսքին 
է դուրս բերում «Ողջո՜ւյն, աշխա՛րհ։» տեքստը։

Գործարկենք տեքստային խմբագրիչը (կամ ծրագրերի մշակման 
ինտեգրացված միջավայրը ― IDE) և նրա օգնությամբ ստեղծենք 
\texttt{hello.cxx} ֆայլը՝ հետևյալ պարունակությամբ․

\begin{Verbatim}
// Առաջին C++ ծրագիրը
#include <iostream>

int main()
{
    std::cout << "Ողջո՜ւյն, աշխա՛րհ։\n";
}
\end{Verbatim}

Ծրագրի տեքստը ֆայլում հիշելուց հետո այն պետք է C++ 
\emph{կոմպիլյատորի} (compiler) օգնությամբ \emph{թարգմանել}
(compile) մեքենայական հրամանների և \emph{կատարել} (execute)։ 
Բայց, մինչև այս քայլերի մանրամասն նկարագրությունը, մի փոքր 
առաջ անցնենք ու ծրագրի ամեն մի տողի համար տանք համառոտ 
մեկնաբանություն։

Առաջին տողը \verb|//| նիշերով սկսվող և մինչև տողի վերջը շարունակվող 
\emph{մեկնաբանություն} է։ Մեկնաբանությունները, որոնք կարող են 
պարունակել կամայական տեքստ, ծրագրի իմաստի վրա չեն ազդում և ամբողջովին 
անտեսվում են կոմպիլյատորի կողմից։ Դրանք ծառայում են մարդ-ընթերցողին 
ծրագրի մասին լրացուցիչ տեղեկություններ տրամադրելու համար։

Երկրորդ տողում գրված \verb|#include| հրահանգը \emph{նախապրոցեսորից} 
(preprocessor) պահանջում է ծրագրի տեքստին կցել \texttt{iostream} 
գրադարանային ֆայլի պարունակությունը։ Այս ֆայլում են սահմանված ներմուծման 
ու արտածման ստանդարտ հոսքերի հետ աշխատող գործողություններն ու օբյեկտները։ 
Ի թիվս զանազան ֆունկցիաների ու օբյեկտների, \texttt{iostream} ֆայլը 
պարունակում է արտածման ստանդարտ հոսքի հետ կապված \texttt{cout} օբյեկտը, 
և իր աջ արգումենտի արժեքը ձախ արգումենտով տրված հոսքի մեջ ուղարկող 
\verb|<<| գործողությունը։ C++20 ստանդարտում ներմուծվել է \emph{մոդուլի} 
(module) գաղափարը, որի մասին կխոսենք \verb|??| գլխում։

C++ լեզվով գրված ամեն մի ծրագիր պետք է պարունակի \texttt{main} անունով 
ֆունկցիա։ Այս ֆունկցիան ծրագրի \emph{մուտքի կետն} (entry point) է. 
ծրագրիրը սկսում է կատարվել \texttt{main} ֆունկցիայից։ Ավելի ճիշտ ասած՝ 
ծրագիրը կատարելու համար \emph{օպերացիոն համակարգը} «կանչում» է 
\texttt{main} ֆունկցիան։ Տվյալ դեպքում \texttt{main} ֆունկցիան պարամետրեր 
չունի և վերադարձնում է \texttt{int} (integer, ամբողջ թիվ) տիպի արժեք։ 
UNIX/Linux օպերացիոն համակարգերն այս ֆունկցիայի արժեքը մեկնաբանում են 
որպես ծրագրի հաջող կամ անհաջող ավարտի ազդանշան։ Ըստ պայմանավորվածության, 
\texttt{0} արժեքը համարվում է հաջող ավարտ, իսկ \texttt{0}֊ից տարբեր դրական
ամբողջ թվերը՝ անհաջող ավարտ։ C++ ֆունկցիայից որևէ արժեք է վերադարձվում է 
\texttt{return} հրամանով։ Սակայն միայն \texttt{main} ֆունկցիայի համար կա 
բացառություն․ եթե բացահայտորեն նշված չէ \texttt{return} հրամանը, ապա 
վերադարձվող արժեքը համարվում է \texttt{0}։

\texttt{main} ֆունկցիայի \emph{մարմինը}, որ պարփակված է \verb|{| և 
\verb|}| փակագծերի մեջ, պարունակում է միակ արտահայտություն, որում 
\verb|<<| գործողության միջոցով «\texttt{Ողջո՜ւյն, աշխա՛րհ։}» տողն 
ուղարկվում է արտածման ստանդարտ հոսքի հետ կապված \texttt{cout} օբյեկտին։ 
\verb|::| գործողությամբ \texttt{cout} անունին կցված \texttt{std} նախդիրը 
ցույց է տալիս, որ այդ երկու անունները պատկանում են լեզվի \emph{Կաղապարների 
ստանդարտ գրադարանի} (Standard template library, STL) անունների տիրույթին։ 
(Ֆունկցիաների հայտարման, սահմանման և կիրառման մասնին մանրամասնորեն խոսում 
ենք \verb|??| գլխում։)

Հիմա այս ծրագիրը թարգմանենք՝ \emph{կոմպիլյացնենք}, մեքենայական կոդի և կատարենք։

C++ լեզվով գրված ծրագիրը թարգմանության ժամանակ անցնում է երեք հիմնական փուլ.
\begin{itemize}
\item \emph{նախնական մշակում} (preprocessing) -- մշակվում են \verb|#| 
նիշով սկսվող հրահանգները, որոնք ծրագրի հետ կատարում են զուտ տեքստային 
գործողություններ։ Օրինակ, ֆայլերի կցում, անունների փոխարինում արժեքներով 
և այլն,
\item \emph{թարգմանություն} (compilation) -- ծրագրավորողի գրած տեքստը 
թարգմանվում է մեքենայական կոդի և ստեղծվում է \emph{օբյեկտային} մոդուլ,
\item \emph{կապակցում} կամ \emph{կապերի խմբագրում} (linking) -- ստեղծված 
օբյեկտային ֆայլերը կապակցվում են գրադարանային (կամ այլ) օբյեկտային ֆայլերի 
հետ և կառուցվում է \emph{կատարվող մոդուլ}։
\end{itemize}

Սովորաբար ծրագրի տեքստից կատարվող մոդուլի ստացման այս երեք փուլերը 
թաքնված են օգտագործողից և, պարզ ծրագերի դեպքում, կատարվում են մեկ 
հրամանով։ Օրինակ, Bash ինտերպրետատորի հրամանային տողում ներմուծենք 
հետևյալը.

\begin{Verbatim}
$ c++ hello.cxx
\end{Verbatim}

Եթե կոմպիլյատորը որևէ սխալ չի հայտնաբերել, ապա ընթացիկ պանակում ստեղծվում 
է \texttt{a.out} անունով կատարվող մոդուլը։ Եթե հարկավոր է կատարվող մոդուլին 
այլ անուն տալ, ապա այն պետք է նշել որպես \texttt{c++} հրամանի \texttt{-o} 
պարամետրի արժեք։ Օրինակ.

\begin{Verbatim}
$ c++ -o hello hello.cxx
\end{Verbatim}

\noindent այս հրամանի հաջող կատարման արդյունքում կատարվող մոդուլը, 
\texttt{a.out}֊ի փոխարեն, ստանում է \texttt{hello} անունը։

\texttt{hello} մոդուլն աշխատեցնելու համար պետք է Bash֊ի հրամանային 
տողում ներմուծել.

\begin{Verbatim}
$ ./hello
Ողջո՜ւյն, աշխա՛րհ։
\end{Verbatim}

Ծրագրի կատարման արդյունքում տեսնում ենք արտածված տողը. սա հենց այն է, 
ինչ ակնկալում էինք առաջին ծրագրից։

Որպեսզի համոզվենք, որ իսկապես \texttt{main} ֆունկցիայի վերադարձրած արժեքը 
փոխանցվում է օպերացիոն համակարգին, ծրագրի կատարումից հետո Bash ինտերպրետատորի 
\texttt{echo} հրամանով դուրս բերենք \verb|$?| փսևդոփոփոխականի արժեքը։

\begin{Verbatim}
$ echo $?
0
\end{Verbatim}

Հիմա վերը բերված օրինակի \texttt{hello.cxx} ֆայլը պատճենենք 
\texttt{hello2.cxx} անունով, և \texttt{main} ֆունկցիայի մարմինն ավարտող 
\verb|}| փակագծից առաջ ավելացնենք «\texttt{return 7;}» հրամանը։

\begin{Verbatim}
#include <iostream>

int main()
{
    std::cout << "Ողջո՜ւյն, աշխա՛րհ։\n";
    return 7;
}
\end{Verbatim}

Այնուհետև նորից թարգմանենք ծրագիրը, աշխատեցնենք ու արտածենք \verb|$?|
փսևդոփոփոխականի արժեքը։ Հրամանների հաջորդականությունը և արտածված 
արդյունքները հետևյալ տեսքն ունեն.

\begin{Verbatim}
$ c++ -o hello2 hello2.cxx
$ ./hello2
Ողջո՜ւյն, աշխա՛րհ։
$ echo $?
7
\end{Verbatim}


