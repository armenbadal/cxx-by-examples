\chapter{Գնդի մակերեսը, ծավալը և զանգվածը}

Դիտարկենք պողպատե սնամեջ մի գունդ՝ \texttt{diameter} տրամագծով 
և մետաղի \texttt{thickness} հաստությամբ։ Պետք է գրել ծրագիր, որը 
հաշվում է այդ գնդի մակերևույթի \emph{մակերեսը}, խոռոչի \emph{ծավալը} 
և դատարկ գնդի \emph{զանգվածը}։

Օգտագործողից պահանջվում է ներմուծման ստանդարտ հոսքից տալ տրամագծի 
ու մետաղի հաստության արժեքները, իսկ ծրագրում պետք է հաշվել ու արտածման 
ստանդարտ հոսքին արտածել պահանջվող արդյունքները։

Ինչպես և նախորդ գլխում, ամբողջ ծրագիրը գրելու ենք \texttt{main()} 
ֆունկցիայում։

\begin{Verbatim}
int main()
{
  // ... 
}
\end{Verbatim}

Հաշվարկների միջանկյալ արժեքները պահելու համար ծրագրավորման լեզուներում 
օգտագործում են \emph{փոփոխականի} գաղափարը։ Այն մի \emph{օբյեկտ} է, որը 
ծրագրի կատարման ընթացքի տարբեր պահերին կարող է ընդունել տարբեր արժեքներ։ 
Ծրագրի տեքստում փոփոխականը սահմանվում է իր \emph{տիպն} ու \emph{անունը} 
նշելով։ Օրինակ, տրամագծի արժեքը պահելու համար կարող ենք սահմանել․

\begin{Verbatim}
double diameter = 0.0;  // տրամագիծ
\end{Verbatim}

Այստեղ \emph{double}֊ը տիպն է, որը նշանակում է \emph{կրկնակի ճշտության 
իրական թիվ}, \texttt{diameter}֊ը փոփոխականի անունն է, իսկ \texttt{0.0}֊ն՝ 
սկզբնական արժեքն է։ 

Ծրագրի կատարման ժամանակ յուրաքանչյուր փոփոխականի հատկացվում է հիշողության 
մի որոշակի հատված։ Այդ հատվածի չափը որոշվում է փոփոխականի տիպով։ Օրինակ, 
սովորաբար \texttt{double} տիպին հատկացվում է 8 բայթ (4 բայթ՝ 32 բիթանոց 
ճարտարապետությունների դեպքում)։ Փոփոխականին հատկացված հիշողության տիրույթի 
առաջին բայթի համարը կոչվում է փոփոխականի (օբյեկտի) \emph{հասցե}։ Օրինակ, 
ենթադրենք, թե մեր ծրագրում սահմանված են \texttt{diameter} իրական և 
\texttt{count} ամբողջաթիվ փոփոխականները․

\begin{Verbatim}
double diameter = 0.0;
int count = 0;
\end{Verbatim}

Հիշողության մեջ դրանց տեղավորման հնարավոր տարբերակ կարող է լինել հետևյալը․

%![նկար]()

Այսինքն՝ կատարման ժամանակ \texttt{diameter}֊ին հատկացվել է 1000֊րդ բայթից 
սկսվող ութ բայթանոց տիրույթը, իսկ \texttt{count}֊ին՝ 2000֊րդ բայթից սկսվող 
չորս բայթանոց տիրույթը։

C++ լեզուն թույլ է տալիս \verb|&| միտեղանի նախածանցային գործողությամբ 
ստանալ փոփոխականի հասցեն։ Օրինակ, \texttt{count}֊ի հասցեն կարող ենք արտածել 
հետևյալ արտահայտությամբ․

\begin{Verbatim}
std::cout << &count << std::endl;
\end{Verbatim}

Հասցեի համար նույնպես կարելի է փոփոխական սահմանել։ Պարզապես պետք է 
համապատասխան տիպի անունից հետո գրել \verb|*| (աստղանիշ) նշանը։ Օրինակ, 
հետևյալ հրամանով․

\begin{Verbatim}
int* count_p = &count;
\end{Verbatim}

\noindent սահմանվում է \texttt{int} տիպի օբյեկտի հասցե պարունակող 
\texttt{count\_p} փոփոխականը և այն արժեքավորվում է \texttt{count}֊ի 
հասցեով։ Հասցեի արժեք պահելու համար նախատեսված փոփոխականը կոչվում է 
\emph{ցուցիչ} (pointer)։ Եթե հարկավոր է նշել, որ ցուցիչը ոչ մի 
փոփոխականի ցույց չի տալիս, ապա դրան վերագրում ենք \texttt{nullptr} 
համապիտանի արժեքը։

Մեր ծրագիրը պետք է օգտագործողին առաջարկի ներմուծել գնդի տրամագծի 
արժեքը, ապա ներմուծման ստանդարտ հոսքից \texttt{diameter} փոփոխականի 
մեջ կարդա ներմուծված արժեքը։ Այսպես․

\begin{Verbatim}
std::cout << "Գնդի տրամագիծը (մ). ";
double diameter = 0.0;
std::cin >> diameter;
\end{Verbatim}

Առաջին տողով արտածվում է ներմուծման հրավերքը, երկրորդ տողում սահմանված 
է \texttt{0.0}֊ով սկզբնարժեքավորված \texttt{diameter} փոփոխականը, իսկ 
երրորդ տողում \texttt{diameter}֊ի՝ ներմուծման հոսքից կարդալու 
արտահայտությունն է։ Նկատենք, որ \verb|std::cin >> ...| արտահայտությունը 
սիմետրիկ է \verb|std::cout << ...| արտահայտությանը։ Երկու դեպքերում էլ 
\verb|<<| (կամ \verb|>>|) գործողությունը ցույց է տալիս տվյալների 
«շարժման ուղղությունը»։ Եթե արտածման դեպքում տվյալներն ուղղվում են 
դեպի \texttt{std::cout} օբյեկտով որոշվող արտածման հոսքին, ապա ներմուծման 
դեպքում տվյալները \texttt{std::cin} օբյեկտով որոշվող ներմուծման հոսքից 
ուղղվում են դեպի համապատասխան փոփոխականը։

Ճիշտ նույն կերպ կազմակերպում ենք նաև պատի հաստության արժեքի ներմուծումը․

\begin{Verbatim}
cout << "Գնդի պատի հաստությունը (մ). ";
double thickness = 0.0;
cin >> thickness;
\end{Verbatim}

Գնդի մակերևույթի մակերեսը հաշվելու համար օգտագործելու ենք հայտնի բանաձևը․

\[
S = 4\pi r^2
\]

\noindent որտեղ \(r\)֊ը գնդի շառավիղն է, իսկ \(\pi\)-ն՝ հայտնի 
մաթեմատիկական հաստատունը։

Շառավիղը տրամագծի կեսն է։ Հաշվենք այն ու վերագրենք \texttt{radius}֊ին․

\begin{Verbatim}
const auto radius = diameter / 2;
\end{Verbatim}

\texttt{const} բառը ցույց է տալիս, որ \texttt{radius}֊ը սահմանված է 
որպես հաստատուն և նրան տրված սկզբնական արժեքը ծրագրի կատարման ընթացքում 
չի փոխվելու։

Տիպի փոխարեն գրված \texttt{auto} բառը C++ լեզվի կոմպիլյատորից 
«պահանջում» է ինքնուրույն դուրս բերել \texttt{radius}֊ի տիպը՝ 
հիմնվելով դրան տրված արժեքին։ Տվյալ դեպքում, քանի որ 
\texttt{diameter}֊ի տիպը \texttt{double} է, բաժանման արդյունքը 
նույնպես կստացվի \texttt{double}, հետևաբար \texttt{radius}֊ի տիպը 
դուրս է բերվում նորից որպես \texttt{double}։ 

Մակերեսը հաշվելու բանաձևում մասնակցող \(\pi\) մեծության արժեքն էլ 
կարելի է սահմանել որպես հաստատուն․

\begin{Verbatim}
const double PI = 3.14159;
\end{Verbatim}

Սակայն ավելի ճիշտ է օգտագործել C++ լեզվի C++20 ստանդարտում սահմանված 
արժեքը։ Այն \texttt{<numbers>} գրադարանից հասանելի է 
\texttt{std::numbers::pi} անունով։

Վերջապես հաշվենք ու արտածենք պահանջվող արժեքը․

\begin{Verbatim}
const auto surface_area = 4 * pi * radius * radius;
std::cout << "Գնդի մակերևույթի մակերեսը (մ²). "
          << surface_area << std::endl;
\end{Verbatim}

Այստեղ տվյալների արտածումը կազմակերպված է \verb|<<| գործողություններով 
իրար կապված շղթայի տեսքով։ Նախ՝ \texttt{std::cout}֊ին է ուղղարկվում 
«\texttt{Գնդի մակերևույթի մակերեսը (մ²). }» տողը, ապա՝ 
\texttt{surface\_area} փոփոխականի արժեքը, և վերջում՝ \texttt{std::endl}
\emph{մանիպուլյատորը}։ Վերջինս արտածման հոսքի վերջից ավելացնում է նոր 
տողի անցման նիշը։

Հաջորդը պետք է հաշվենք գնդի խոռոչի ծավալը՝ նորից կիրառելով հետևյալ 
հայտնի բանաձևը, որում \(r_i\)֊ն խոռոչի շառավիղն է․

\[
S = \frac{4}{3}\pi r_i^3 
\]

Դրա համար նախ հաշվենք խոռոչի շառավիղը՝ գնդի շառավղից հանելով 
պատի հաստությունը․

\begin{Verbatim}
const auto inner_radius = radius - thickness;
\end{Verbatim}

\begin{Verbatim}
const auto inner_volume = (4 * pi * inner_radius * inner_radius * inner_radius) / 3;
std::cout << "Գնդի խոռոչի ծավալը (մ³). "
          << inner_volume << std::endl;
\end{Verbatim}
